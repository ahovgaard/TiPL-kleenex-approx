\section{Introduction}

Approximate pattern matching is the problem of finding occurrences of a given
pattern string in a given text string, while allowing some degree of error in
the occurrences. In other words, given some distance metric, the problem is to
find all the occurrences of the pattern in the text including those within a
specified distance of the pattern string. The pattern may denote a single
string, or a set of strings as in the case of a regular expression.

The problem is also known as approximate string matching or searching,
approximate regular string matching, approximate regular substring matching
etc. It is an important problem in many areas, including in bioinformatics
where common tasks involve searching for specific patterns in biological
sequences such as DNA, RNA, and protein, represented as text strings. However
these sequences may contain errors, thus presenting a need for approximate
search techniques.

A number of bachelor theses investigated the use of automata-based techniques
for DNA pattern matching and they compared their implementations to the widely
used domain-specific tool \sfm{}.  This turned out to be just barely
competitive with \sfm{} in some cases. Even more recently, the Kleenex language
and compiler~\cite{grathwohl2016kleenex,soholm2015ordered} was extended to
support approximate matching, and in fact approximate transduction, through
rewriting of core Kleenex programs to express approximate matching as exact
matching~\cite{troelsen2016approximate}. This has indicated even better
performance and even managed to outperform \sfm{} in one case. The evaluation
of this tool with regards to the application of approximate pattern matching in
biological sequences has, however, only been briefly investigated.

This project is about investigating the applicability of Kleenex to the topic
discussed above and comparing it to the NR-grep~\cite{navarro2001nr} program as
well as \sfm{}.

We have found that approximate Kleenex shows a competitive runtime performance
in many case, however often slightly outperformed by NR-grep, and while it is
sometimes outperforming \sfm{}, it does not seem to scale as well with increasing
number of allowed errors $k$. We have also found that the size of the program
generated by approximate Kleenex grows very large as $k$ increases, which is
possibly a major cause of performance degradation.

We also review other existing solutions to the problem of approximate matching
and we discuss potential improvements to the approximate Kleenex program.

%%% Local Variables:
%%% mode: latex
%%% TeX-master: "main"
%%% End:
