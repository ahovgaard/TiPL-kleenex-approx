\section{Background}

This section describes the main background theory of the Kleenex language and
its application to approximate string matching.

\subsection{Transducers}

Kleenex is a domain-specific language for expressing transducers. The concept
of a transducer extends that of a finite automaton to also include output, that
is, a finite automaton either accepts or rejects a string, whereas a transducer
produces an output string in a language over a given output alphabet.

First we define the notion of a finite state transducer, which is essentially
just a nondeterministic finite automaton (NFA) which can also output a symbol
on each transition.

\begin{definition}[FST]
  A \emph{finite state transducer} $\mathcal{T}$ over an input alphabet
  $\Sigma$ and an output alphabet $\Gamma$ is a structure
  $(\Sigma, \Gamma, Q, q^s, q^f, \Delta)$, where
  \begin{itemize}
      \item $Q$ is a finite set of states,
      \item $q^s, q^f \in Q$ are the initial and final states,
        respectively, and
      \item $\Delta \subseteq Q \times \Sigma[\epsilon] \times
        \Gamma[\epsilon] \times Q$ is the transition relation.
  \end{itemize}
\end{definition}

Here, $\Sigma[\epsilon]$ denotes the alphabet $\Sigma$ extended with the empty
string $\epsilon$.

Intuitively, there is the following correspondence between the elements of the
transition relation and the transitions in the drawn diagram of the transducer:
\[
  (q, s, t, q') \in \Delta \quad \Leftrightarrow \quad q \xrightarrow{s/t} q'
\]

In an NFA, multiple paths may lead to an accepting state for a given input
string, but regardsless of which accepting path is chosen, the ouput is the
same. In an FST, on the other hand, the path determines the transduced output,
thus the same input string may produce entirely different outputs depending on
the path chosen.

In order to disambiguate between different transductions of the same input
string, we introduce the notion of a normalized FST. To facilitate this, we
also introduce two new symbols $\epsilon_0$ and $\epsilon_1$. Both denote the
empty string, but they serve to distinguish between nondeterministic
$\epsilon$-transitions.

\begin{definition}[NFST]
  A \emph{normalized finite state transducer} is a deterministic FST,
  $T = (\Sigma[\epsilon_0,\epsilon_1], \Gamma, q^s, q^f, \Delta)$, with the
  constraint that $\forall q \in Q$, $q$ is either a choice state, a skip
  state, a symbol state, or the final state.
\end{definition}

The notions of choice, skip, or symbol states are perhaps best conveyed by
illustration, as shown in Figure~\ref{fig:nfst-states}.

\begin{figure}[!ht]
  \centering
  \begin{subfigure}[b]{0.3\textwidth}
    \centering
    \begin{tikzpicture}[->,>=stealth',shorten >=1pt,auto,node distance=1.0cm,
      semithick, main/.style={circle,draw,minimum width=10pt}]
        \node[main] (q0) {$ $};
        \node       (q1) [right = of q0] {};
        \node       (q2) [below right = of q0] {};
        \path (q0) edge node {$\epsilon_0/\cdot$} (q1)
              (q0) edge node [below left = -2mm, swap]
                   {$\epsilon_1/\cdot$} (q2);
    \end{tikzpicture}
    \caption{A choice state.}
  \end{subfigure}
  ~
  \begin{subfigure}[b]{0.3\textwidth}
    \centering
    \begin{tikzpicture}[->,>=stealth',shorten >=1pt,auto,node distance=1.0cm,
        semithick, main/.style={circle,draw,minimum width=10pt}]
        \node[main] (q0) {$ $};
        \node       (q1) [right = of q0] {};
        \path (q0) edge node        {$\epsilon/\cdot$} (q1);
    \end{tikzpicture}
    \caption{A skip state.}
  \end{subfigure}
  ~
  \begin{subfigure}[b]{0.3\textwidth}
    \centering
    \begin{tikzpicture}[->,>=stealth',shorten >=1pt,auto,node distance=1.0cm,
      semithick, main/.style={circle,draw,minimum width=10pt}]
        \node[main] (q0) {$ $};
        \node       (q1) [right = of q0] {};
        \path (q0) edge node {$a/\cdot$} (q1);
    \end{tikzpicture}
    \caption{A symbol state.}
  \end{subfigure}
  \caption{Illustration of choice states, skip states, and symbol states.}
  \label{fig:nfst-states}
\end{figure}

Thus, the set of symbols that any state has transitions on, i.e. the support of
the state, is either just $\epsilon$, exactly $\{\epsilon_0, \epsilon_1\}$,
exactly one $a \in \Sigma$, or $\emptyset$ in the case of the final
state. % Therefore, an NFST is deterministic...





%%% Local Variables:
%%% mode: latex
%%% TeX-master: "main"
%%% End:
