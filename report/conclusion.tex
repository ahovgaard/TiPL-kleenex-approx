\section{Conclusion}
Our experiments show that as long as the patterns are simple enough and that
error measure $k$ is low, so that the executables that Kleenex generates are
not too large, it seems to be able to compete with or outperform the other
tools we have tested, with NR-grep being the biggest difference, but it does
have an unfair advantage in using the input with newlines. However the compile
times are quite unpractical with some pattern needing hours to compile, while
it is also impossible to compile even fairly simple patterns to any more than
$k = 3$, it serverly limits to real world pracitcality of approximate Kleenex
in its current form. But if the SST size complexity with regards to $k$ could
be reduced, in one or more ways, so as to bring down compile times and also
executable size, it seems likely that approximate Kleenex could be a useful
tool for doing high throuhput approximate string matching.
