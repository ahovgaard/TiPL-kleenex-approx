\section{Problem analysis}

\subsection{Analysis of automata generated by approximate kleenex}

Visualization.

\subsection{Improving kleenex}

% Counters, implemented as optimization in the implementation vs. updating the
% automata models.

In his bachelor's thesis~\cite{enevoldsen2015pattern}, Sune Enevoldsen uses the
concept of a tagged automaton, based on work by Ville
Laurikari~\cite{laurikari2000nfas, laurikari2001efficient} and inspired by
Levenshtein automata~\cite{schulz2002fast}, to do approximate regular string
matching.

A tagged NFA is an NFA extended with a set of tags which may be set or modified
on each transition in the transition relation. The idea is to use these tags as
counters which keep track of the allowed number of errors, e.g.  insertions,
deletion, and substitutions.

% Thus, the tagged automata for a given regular expression $RE$ will accept the
% strings in $\mathcal{L}(RE)$ as well as any string that is within a certain
% error distance of a string in $\mathcal{L}(RE)$.






%%% Local Variables:
%%% mode: latex
%%% TeX-master: "main"
%%% End:
